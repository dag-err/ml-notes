\documentclass{article}

\title{Complexity Measures}
\author{Vladimir Feinberg}

\usepackage{nicefrac}
\usepackage[hidelinks]{hyperref}
\usepackage{xcolor}
\usepackage{url}
\usepackage{lmodern}
\usepackage{amsmath}
\usepackage{amsthm}
\usepackage{amssymb}
\usepackage{amsfonts}
\usepackage{enumerate}
\usepackage{graphicx}
\usepackage{fullpage}
\usepackage{bbm}
\usepackage{caption}
\usepackage{mathrsfs}
\usepackage[all]{xy}

\newcommand{\pa}[1]{ \left({#1}\right) }

\def \N {\mathbb{N}}
\def \Nbr {\mathcal{N}}
\def \Q {\mathbb{Q}}
\def \F {\mathbb{F}}
\def \then {\implies &}
\def \oif {\Longleftrightarrow &\,}
\def \given {\text{Given }&}
\def \assume {\text{Assume }&}
\def \thfr {\therefore &\enskip}
\def \bij {\leftrightarrow}
\def \inj {\rightarrowtail}
\def \sur {\twoheadedrightarrow}
\def \Z {\mathbb{Z}}
\def \R {\mathbb{R}}
\def \C {\mathbb{C}}
\def \T {\mathbb{T}}
\def \iff {\Longleftrightarrow}
\def \kron {\boldsymbol\delta}
\def \indicator {\mathbbm{1}}

\def\Ta{\textbf{a}}
\def\Tm{\textbf{m}}
\def\Tb{\textbf{b}}
\def\Tc{\textbf{c}}
\def\Td{\textbf{d}}
\def\Te{\textbf{e}}
\def\Tv{\textbf{v}}
\def\Tx{\textbf{x}}
\def\Tw{\textbf{w}}
\def\Ty{\textbf{y}}
\def\Tk{\textbf{k}}
\def\Tt{\textbf{t}}
\def\Tz{\textbf{z}}
\def\Tl{\textbf{l}}
\def\quotient{\mathclose{}/\mathopen{}}
\def\Tf{\textbf{f}}
\def\Th{\textbf{h}}
\def\Tg{\textbf{g}}
\def\sumn{\sum_{n=0}^\infty}
\def\limn{\lim_{n\rightarrow\infty}}
\def\prodn{\prod_{n=0}^\infty}

\def\bsz{\textbf{0}}
\def\bs1{\textbf{1}}
\def\bsa{{\boldsymbol\alpha}}
\def\bse{{\boldsymbol\eta}}
\def \bss{ {\boldsymbol\sigma}}
\def \bsth{ {\boldsymbol\theta}}

\DeclareMathOperator{\conv}{conv}
\DeclareMathOperator\cat{cat}
\DeclareMathOperator\adj{adj}
\DeclareMathOperator{\Laplace}{Laplace}
\DeclareMathOperator\Poisson{Poisson}
\DeclareMathOperator\Id{Id}
\DeclareMathOperator\Uniform{Uniform}
\DeclareMathOperator\mathProb{\mathbb{P}}
\def\P{\mathProb} % need to overwrite stupid paragraph symbol
\DeclareMathOperator\mathExp{\mathbb{E}}
\def \E {\mathExp}


\newcommand{\stc}[1]{\widetilde{#1}}   
\newcommand{\set}[2]{ \left\{ #1 \,\middle|\, #2 \right\} }
\newcommand{\idx}[3]{ \left\{ #1 \right\}_{ #2 }^{ #3 } }
\newcommand{\card}[1]{\left\vert{#1}\right\vert}
\newcommand{\colv}[1]{\begin{pmatrix} #1 \end{pmatrix}}
\newcommand{\mat}[1]{\begin{pmatrix} #1 \end{pmatrix}}
\newcommand{\detmat}[1]{\begin{vmatrix} #1 \end{vmatrix}}
\newcommand{\spanb}[1]{\text{span}\{ #1 \}}
\newcommand{\abs}[1]{\left|#1\right|}
\newcommand{\opcat}[1]{#1^{\text{op}}}
\newcommand{\Inner}[1]{\left\langle #1 \right\rangle}
\newcommand{\Innercpy}[1]{\Inner{ #1, #1 }}
\newcommand{\norm}[1]{\left\| #1 \right\|}% use instead of $\|x\|$

\def\eqd{\mathrel{\overset{\Delta}{=}}}

\DeclareMathOperator{\diag}{diag}
\DeclareMathOperator{\vcdim}{VC-dim}
\DeclareMathOperator*{\Err}{\text{err}}
\DeclareMathOperator*{\ErrE}{\mathbb{E}}
\DeclareMathOperator{\Tr}{tr}
\DeclareMathOperator{\Dim}{dim}
\DeclareMathOperator{\Rank}{rank}
\DeclareMathOperator*{\argmin}{argmin}
\DeclareMathOperator*{\proj}{proj}
\DeclareMathOperator{\Ker}{ker}
\DeclareMathOperator{\Diam}{diam}
\DeclareMathOperator{\Int}{int}
\DeclareMathOperator{\Clo}{clo}
\DeclareMathOperator{\sgn}{sgn}
\DeclareMathOperator{\MyRe}{Re}
\DeclareMathOperator{\MyIm}{Im}
\DeclareMathOperator{\image}{image}
\DeclareMathOperator{\colim}{colim}
\DeclareMathOperator{\Supp}{supp}
\DeclareMathOperator{\Var}{var}
\DeclareMathOperator{\Hom}{hom}
\DeclareMathOperator{\Ob}{ob}
\DeclareMathOperator{\El}{el}
\DeclareMathOperator\power{{\mathcal{P}}}
\DeclareMathOperator{\Nat}{Nat}
\DeclareMathOperator{\cone}{cone}
\DeclareMathOperator{\vectorize}{vec}
\DeclareMathOperator{\matricize}{mat}

% probability stuff
\def \sa {{$\sigma$-algebra}}
\def\OR{{\overline{\R}}}
\def\OX{{\overline{X}}}

\def\mcU{{\mathcal{U}}}
\def \mcX {\mathcal{X}}
\def \mcS {\mathcal{S}}
\def \mcY {\mathcal{Y}}
\def \mcH {\mathcal{H}}
\def \mcD {\mathcal{D}}
\def \mcC {\mathcal{C}}
\def \mcA {\mathcal{A}}
\def \mcK {\mathcal{K}}
\def \mcM {\mathcal{M}}
\def\mcG{{\mathcal{G}}}
\def\mcH{{\mathcal{H}}}
\def\mcF{{\mathcal{F}}}
\def\mcB{{\mathcal{B}}}
\def\mcE{{\mathcal{E}}}
\def\mcI{{\mathcal{I}}}
\def\mcQ{{\mathcal{Q}}}
\def\mcM{{\mathcal{M}}}
\def\mcC{{\mathcal{C}}}
\def\mcT{{\mathcal{T}}}
\def\mcO{{\mathcal{O}}}
\def\mcJ{{\mathcal{J}}}

\makeatletter
\DeclareFontFamily{U}  {MnSymbolF}{}
\DeclareSymbolFont{symbolsMN}{U}{MnSymbolF}{m}{n}
\SetSymbolFont{symbolsMN}{bold}{U}{MnSymbolF}{b}{n}
\DeclareFontShape{U}{MnSymbolF}{m}{n}{
    <-6>  MnSymbolF5
   <6-7>  MnSymbolF6
   <7-8>  MnSymbolF7
   <8-9>  MnSymbolF8
   <9-10> MnSymbolF9
  <10-12> MnSymbolF10
  <12->   MnSymbolF12}{}
\DeclareFontShape{U}{MnSymbolF}{b}{n}{
    <-6>  MnSymbolF-Bold5
   <6-7>  MnSymbolF-Bold6
   <7-8>  MnSymbolF-Bold7
   <8-9>  MnSymbolF-Bold8
   <9-10> MnSymbolF-Bold9
  <10-12> MnSymbolF-Bold10
  <12->   MnSymbolF-Bold12}{}
\DeclareMathSymbol{\tbigtimes}{\mathop}{symbolsMN}{2}
\newcommand*{\bigtimes}{%
  \DOTSB
  \tbigtimes
  \slimits@ 
}
\makeatother

% category theory arguments
% more equality: http://tex.stackexchange.com/questions/333629/
\newcommand\superequiv{ \mathrel{\rlap{\raisebox{\fontdimen22\textfont2}{$ = $}}\raisebox{-0.5\fontdimen22\textfont2}{$ = $}}}
\newcommand{\catfst}{{-}}
\newcommand{\catsnd}{{=}}
\newcommand{\cattrd}{{\equiv}}
\newcommand{\catfth}{{\superequiv}}


\newcommand{\nurl}[2]{\href{ #1 }{\color{blue}\underline{#2}}}

\begin{document}

\maketitle

Complexity measures evaluate the expressiveness of a hypothesis class; they are useful to the extent with which they relate sample and generalization error.

\section{Setup}

We suppose that our data comes in the form of ordered pairs from $\mcX\times \mcY$. Samples follow a particular distribution $(x, y)\sim D$. A hypothesis class $\mcH$ is set of functions $\mcX\rightarrow\mcY$.

A common approach to supervised learning is ERM, where $m$ iid samples from $D$, $S$, are used to find the $h\in\mcH$ minimizing a specified loss $\ell:\mcY^2\rightarrow\R$ over this set. Complexity measures then let us quantify exactly how much loss we can expect when sampling from $D$ again.

We seek to quantify the generalization gap with the help of our notions of complexity. For a fixed $h\in \mcH$:

$$
\varepsilon= \E\left[\ell\pa{h(x), y)}|(x,y)\sim D\right]-\E\left[\ell\pa{h(x), y)}|(x,y)\sim \Uniform(S)\right]
$$

Analysis of Rademacher complexity is agnostic to $h,\ell$; the hypothesis class might as well consist of functions $g:\mcX\times\mcY\rightarrow\R$ yielding their composition. VC dimension analysis, however, requires $\mcY=\{0, 1\}$ and $\ell(a, b)=\indicator\{a=b\}$. VC dimension is still useful for regression problems, by thresholding hypotheses $h\mapsto \indicator{h>\beta}$ for fixed $\beta$.\footnote{\url{https://stats.stackexchange.com/questions/140430}}


Thus, it is useful to find bounds on $\varepsilon$, the difference between the generalization loss $\E\left[\ell\pa{h(x), y)}\right]$, where $(x,y)\sim D$, and sample loss, where the loss is the expectation before taken for $(x,y)$ is uniform over $S$.

Let the gap between the generalization and sample error be $\varepsilon$.

\section{Complexity Measures}

The empirical Rademacher complexity $\hat{R}_S$ assumes a fixed sample $S$ from $D^m$. It relates complexity of a function class $\mcG$ containing vectorized functions $g\in \mcG$ which take elements $z_i=(x_i, y_i)$ in $S$ and return costs through the correlation of $\mcG$ with noise. Let $\vsigma\sim \Uniform\pa{\pm 1}^m$. Rademacher complexity is then the average empirical one.
$$
\hat{R}_S(\mcG)=\E_\vsigma \sup_g \frac{1}{m}\sum_{i=1}^m{g(z) \sigma_i},\;\;\; R_m(\mcG)=\E_S\hat{R}_S(\mcG)
$$

VC dimension accomplishes a similar task for binary classification by rating the complexity of a hypothesis class $\mcH$. Let hypotheses $\mcH\ni h:\mcX\rightarrow \mcY=\ca{\pm 1}$ be applied elementwise over a vector of inputs $\vx$. First we define the growth function $\Pi_\mcH:\N\rightarrow\N$, which defines the maximum number of distinctions a hypothesis class can make over all sets of points in the input space:
$$
\Pi_\mcH(m)=\max_{\vx\in\mcX^m}\card{\set{h(\vx)}{h\in\mcH}}
$$
Then the VC dimension of $\mcH$ is then $\max\set{m\in\N}{\Pi_\mcH(m)=2^m}$.

\section{Overview of Results}

Proofs can be found in a \nurl{http://ittc.ku.edu/~beckage/ml800/VC_dim.pdf}{cogent write-up} by Prof. Beckage from the University of Kansas.

\subsection{VC Generalization Bounds}

Upper bound. If $d$ is the VC-dimension of $\mcH$, then for any $D$ wp $1-\delta$:
$$
\varepsilon\le \tilde{O}\pa{\sqrt{\frac{d-\log \delta}{m}}}
$$
The above inequality is random since it depends on $S$, the $D^m$-valued rv. TODO. find source removing tilde?

Agnostic lower bound. We may find a $D$ such that with a fixed nonzero probability (a non-negligible set of candidate samples $S$), the following holds:
$$
\varepsilon\ge \Omega\pa{\sqrt{\frac{d}{m}}}
$$

The above implies that in the common case of agnostic hypothesis learning, where we do not know distribution $D$, VC-dimension is, \textit{up to logarithmic factors, asymptotically efficient} in quantifying the generalization gap.

Realizability. Suppose $D$ is realizable wrt $\mcH$, so that there exists an $f\in\mcH$ such that for almost any $(x,y)$ sampled from $D$, $f(x)=y$. Then all statements above hold but with $\sqrt{\varepsilon}$ instead of $\varepsilon$.


\subsection{Growth Function Bounds}

Sauer's Lemma implies that VC dimension $d$ bounds the growth function: in a graph of the logarithm of the growth function vs $m$, growth is linear since $\Pi_\mcH(n)=n$ for $n\le d$. Then for $n>d$, growth is at most logarithmic, i.e., $\log\Pi_\mcH=O(\log m)$. With Massart's Lemma we have wp $1-\delta$:
$$
\varepsilon\le O\pa{\sqrt{\frac{\log\Pi_\mcH(m)-\log\delta}{m}}}
$$
Since the above would be large if $\log\Pi_\mcH(m)\simeq m$, it is clear why Sauer's Lemma enables the essential relationship between learnability and complexity.

\subsection{Rademacher bounds}

With $R_m$ either the empirical or expected Rademacher complexity over the sample for a given $h,\ell$ we have again wp $1-\delta$:
$$
\varepsilon\le 2R_m+O\pa{\frac{\log\nicefrac{1}{\delta}}{m}}
$$
$R_m$ may be NP-hard to compute, depending on $\mcH$. This tells us Rademacher complexity could only be a useful improvement over VC-bounds, asymptotically, if we have an efficient approximation for the empirical Rademacher complexity or some knowledge of $D$ as required to compote the true Rademacher complexity.

\section{Hardness of Learning}

Rademacher and Gaussian Complexities: Risk Bounds and Structural Results by Bartlett and Mendelson.

\end{document}